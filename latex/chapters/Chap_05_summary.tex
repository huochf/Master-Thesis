\chapter{总结与展望}\label{chap:summary}

\section{全文总结}
本文围绕单视角人-物重建任务展开,针对人-物空间关系建模与预测、对未曾见过的物体的域外泛化、自然场景中人-物重建三方面提出了算法,有效的解决了单视角人-物空间关系中存在的挑战。

首先,在人-物空间关系建模与预测方面,提出了基于人-物偏移量的表征方式,使用人体和物体网格模型表面锚点之间的偏移量来刻画人体和物体之间的相对空间关系,并使用主成分降维的方式构造人-物空间关系的隐式空间。接着使用叠层归一化流模型从给定的图片中提取该空间关系的隐式表达,在后优化过程中,通过人-物偏移量损失和人-物重投影损失微调结果,实验中和先前方法在BEHAVE数据集和InterCap数据集进行了对比,本章所提出的算法在人-物重建方面取得了更高的重建精度和运行效率。

其次,针对未曾见过物体的域外泛化问题,传统基于学习的方法往往只能应用于训练集中见过的物体,无法很好地泛化到和新的形状物体交互上。为了应用该挑战,提出使用物体形状归一化模型将物体的形状映射到统一化的形状空间中,建立在该形状空间中人与物体之间的交互,测试时,那些未曾见过的物体被映射到该形状空间中,具有统一的表达,在训练集中的交互先验通过该形状空间迁移到新的物体上。实验中在CHAIRS数据集上和基于物体RT位姿的方法进行了比较,本章方法具有更好的泛化性能和交互迁移能力。

最后,在自然场景中人-物重建方面,由于自然场景中物体种类和交互类型的多样化,基于学习的方法受到数据集的限制不能应对多样性高的自然场景。为了解决这一问题,提出了一种二维监督的方法从大规模的二维数据中学习三维的人物空间交互先验知识的方法,该方法从二维图片中学习人-物交互的视角分布及其在各个视角下人-物二维关键点的分布,为了训练该网络,使用最近邻算法根据图片之间关键点的集合一致性对图片进行聚类以得到同一交互类型在其它视角下的二维关键点的分布情况。为了验证该算法,构造了自然场景中的数据集WildHOI,该数据集包含丰富的人和8个不同物体在各种场景中和不同物体之间的丰富的交互类型。实验中,在BEHAVE数据集和三维监督方法进行了对比,结果表明,本章所提出的方法即使没有直接使用三维标签监督训练网络也能够达到和三维监督方法近乎相媲美的性能,除室内BEHAVE数据集,还在自然场景的WildHOI数据集进行定性和定量实验,结果表明,本文所提出的方法相较于之前的方法对自然场景更加鲁棒,能够在不借助任何三维人-物相对空间关系的标注以及任何人和物体交互先验知识的前提下重建出合理的人和物体空间关系。

综上所述,本文提出的算法在单视角重建任务中取得了显著效果,为人-物交互关系建模和预测、对未曾见过物体的域外泛化、自然场景中人-物重建等问题提供了新的思路和方法。相信在不久的将来,三维空间中人-物交互将会取得更大的突破和进展,希望本文的研究能够为相关领域的研究工作提供一些有益的启示和参考。

\section{未来展望}

尽管本文所提出的算法在解决单视角人-物重建任务中取得较好效果,但仍然存在一些可以进行进一步改进和完善的方向:
\begin{enumerate}
	\item 首先,在人-物空间关系建模与预测方面,可以探索更灵活和有效的表征方式,例如结合局部和全局信息的表示学习方法,以提高人-物空间关系的建模和预测性能。另外,可以尝试结合目标检测和姿态估计等任务的信息,提升对人体和物体之间关系的理解和预测准确性。
	\item 其次,在对未曾见过的物体的域外泛化问题上,可以考虑引入更多的先验知识,例如利用物体的属性信息、材质信息等,提高模型的泛化能力和交互迁移效果。另外,可以探索使用生成对抗网络等方法进行形状归一化和统一化,以提高对未知物体的重建性能。
	\item 最后,在自然场景中人-物重建方面,可以加强对不同场景和不同交互类型的建模和预测能力,例如引入场景语义分割信息、场景布局信息等,提高模型对自然场景中人-物空间关系的理解和重建效果。同时,可以结合强化学习等方法,提高模型在复杂场景下的迁移和泛化能力,进一步提升人-物重建的性能。
\end{enumerate}

未来可以在改进现有算法的基础上,进一步探索和研究更加高效和稳健的方法,推动相关领域的研究工作取得更大的突破和进展。